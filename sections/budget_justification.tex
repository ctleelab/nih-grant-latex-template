%---------------------------------
% Chktex Suppressions
%---------------------------------
% Suppress 2: Using ~ spaces
% chktex-file 2

% Suppress 3: You should enclose the previous parenthesis with `{}'.
% chktex-file 3

% Suppress 6: No italic correction (`\/') found.
% chktex-file 6

% Suppress 8: Wrong length of dash may have been used.
% chktex-file 8

% Suppress 13: Intersentence spacing (`\@') should perhaps be used.
% //chktex-file 13


% cspell:disable
\documentclass[../main.tex]{subfiles}
\graphicspath{{\subfix{../Figures/}}}
% cspell:enable

\begin{document}
\pagenumbering{gobble}

\section{BUDGET JUSTIFICATION}

\subsection{PERSONNEL (Direct Cost Per Year)}
\begin{itemize}
    \item \textbf{Key Person 1, PD/PI, XXX calendar month effort, with/without salary} 
    Dr. Key Person 1 is (insert titles and distinctions here) and has experience with (provide short justification for involvement with this project, and what roles will be assumed)
    \item \textbf{Key Person 2, Role, XX calendar month effort, with/without salary}
    [Dr.?] Key Person 2 is (insert titles and distinctions here) and has experience with (provide short justification for involvement with this project, and what roles will be assumed) Etc.
\end{itemize}


\Red{The Investigators listed at NIH cap have actual salaries that exceed the current NIH cap of \$[Cap Rate].} 
Salaries escalated at \Red{X\%} for each non-capped personnel for years 2-5.


\subsubsection{Fringe Benefits}



\subsection{EQUIPMENT}
\Red{Detail any equipment needs (>\$5,000 items only. If item is less than \$5,000 please consider materials \& supplies cost)}


\subsection{TRAVEL}
\Red{Detail plans for travel, and describe necessity and benefit to the project.}

\subsection{PARTICIPANT/TRAINEE SUPPORT COSTS}
\Red{Training-type proposals only. If Participant/Trainee Support Costs are required to fulfil the program needs, describe those costs here (itemized). If GSR Remission fees, describe in ``Other Direct Costs'' section. See SF424 for further guidance.}

\begin{itemize}
    \item \textbf{Data Management and Sharing Costs:} \$0
       \item \textbf{Equipment:} \$0
       \item \textbf{Consortium/Contractual Arrangements} \$0
       \item \textbf{Exclusions from F\&A base:} \$0 
   \end{itemize}



\subsection{OTHER DIRECT COSTS}
\Red{Identify direct cost items, detail any calculations used to arrive at a number.}
\subsubsection{Materials and Supplies}
\Red{Itemize and justify all costs that exceed \$1,000. Any costs that are less than 1,000 may be grouped as miscellaneous/consumables/etc but should still be justified. Include calculations whenever necessary if cost is based on established rates.}

\subsubsection{Publication Costs}
Explain how much is requested, and to support how many articles, and in which years.


\subsubsection{Consultant Services} 
Justify any need for consultants here. 
Typically, consultants will charge a fixed rate for their services that includes both their direct and F\&A costs. 
You do not need to report separate direct and F\&A costs for consultants; however, you should report how much of the total estimated costs will be spent on travel. 
Consultants are not subject to the salary cap restriction; however, any consultant charges should meet your institution's definition of ``reasonableness''.


\subsubsection{Computer Services}
Justify the need for computing costs here. The services you include here should be research specific computer services- such as reserving computing time on supercomputers or getting specialized software to help run your statistics. This section should not include your standard desktop office computer, laptop, or the standard tech support provided by your institution. Those types of charges should come out of the F\&A costs.

\subsubsection{Subawards/Consortium/Contractual Costs}
Typically this header is not included as subawards/consortiums will provide their own, separate budget justification. 
If you feel additional justification of the cost is required, a short justification may be provided here. 
This is out of the ordinary, and in most cases not necessary.


\subsubsection{Equipment or Facility Rental/User Fees}
 Not to be confused with Equipment Purchases - typically this category is included to justify off-campus research space rental. Occasionally equipment may be rented from vendors, instead of purchased outright, and those rental costs should be justified here. Costs for rent or use of equipment belonging to a Core, however, should be justified as an “Other Cost” as this is a fee-for-service.


\subsubsection{Alterations and Renovations} A\&R does not include general maintenance projects (normally handled under F\&A) or projects exceeding \$500,000 (considered "construction" projects). A\&R can be used for projects such as altering a room to make space for a new grant-related piece of equipment. If applicable:

Justify basis for costs, itemize by category.

Enter the total funds requested for alterations and renovations. Where applicable, provide the square footage and costs.

If A\&R costs are in excess of \$300,000 further limitations apply and additional documentation will be required.


\subsubsection{Next Generation Networking Communications Fee}
UC San Diego Information and Technology Services (ITS) charges a flat per month fee for services to provide state-of-the-art technology infrastructure and services to the campus community.  
These charges are directly attributable and proportionally applied for the individual(s) included in the proposed budget on the project.  
These costs are not included in the campus' F\&A rate as an indirect cost.  
UC San Diego auditors have determined that it is both equitable and consistent with the OMB Circular 2 CFR 200 provisions on cost allocability that the costs be assigned to FTE on grant and contract funds.  
Accordingly, an allocable portion of these NGN costs are included in this budget as direct project costs.

\subsubsection{General Liability Insurance}
UCSD maintains a self-insurance program to cover the costs of its General Liability Program. 
Charges are made as a percentage of salary costs, and assessed to all extramural funding sources with the exception of support from the federal government, federal flow-through funds and certain contracts and grants from state or local governments. 
The expense is based on dollars of salary support expended and is calculated at \Red{\$XXX}. 
We anticipate this expense will be \Red{\$XXX (\(0.0125 \cdot \$XX\))}.

\subsubsection{Graduate Student Researcher Fee Remission}
GSRs appointed at 25\% time or greater qualify for fee remissions that cover 100\% of the Education, Registration, and UCSHIP fees for the quarter(s) of appointment. 
These fees are assessed as direct costs, are required to be paid from the same account-fund as the salary source. 
The current published rate for Graduate Student Researcher Fees is \Red{\$XXX} annually.


\subsection{Indirect Costs:}

UC San Diego's indirect costs are calculated based on Modified Total Direct Costs (MTDC) as defined in 2 CFR Part 200.68 using Facilities and Administration (F\&A) rates approved by the U.S. Department of Health and Human Services (DHHS).

Rates established by UC San Diego's F\&A rate agreement dated 06/06/2024, and until amended, are as follows:\\
July 1, 2022 to June 30, 2023: 58.0\%\\
July 1, 2023 to June 30, 2024: 58.0\%\\
July 1, 2024 to June 30, 2025: 59.0\%\\
July 1, 2025 to June 30, 2026: 59.0\%\\
July 1, 2026 to June 30, 2027: 60.0\%

\end{document}
