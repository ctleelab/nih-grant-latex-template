
%---------------------------------
% Chktex Suppressions
%---------------------------------
% Suppress 2: Using ~ spaces
% chktex-file 2

% Suppress 3: You should enclose the previous parenthesis with `{}'.
% chktex-file 3

% Suppress 6: No italic correction (`\/') found.
% chktex-file 6

% Suppress 8: Wrong length of dash may have been used.
% chktex-file 8

% Suppress 13: Intersentence spacing (`\@') should perhaps be used.
% //chktex-file 13

% cspell:disable
\documentclass[../main.tex]{subfiles}
\graphicspath{{\subfix{../Figures/}}}   
% cspell:enable

\begin{document}
\pagenumbering{gobble}

\section*{Data Management and Sharing Plan:}

\subsection*{Element 1: Data Type}
\subsubsection*{A. Types and amount of scientific data expected to be generated in the project:}
% Summarize the types and estimated amount of scientific data expected to be generated in the project,
\lipsum[1-3]

\subsubsection*{B. Scientific data that will be preserved and shared, and the rationale for doing so:}
% Describe which scientific data from the project will be preserved and shared and provide the rationale for this decision. 
\lipsum[1-3]

\subsubsection*{C. Metadata, other relevant data, and associated documentation:}
% Briefly list the metadata, other relevant data, and any associated documentation (e.g., study protocols and data collection instruments) that will be made accessible to facilitate interpretation of the scientific data.
\lipsum[1-3]

\subsection*{Element 2: Related Tools, Software and/or Code:}
% State whether specialized tools, software, and/or code are needed to access or manipulate shared scientific data, and if so, provide the name(s) of the needed tool(s) and software and specify how they can be accessed.
\lipsum[1-3]

\subsection*{Element 3: Standards}
% State what common data standards will be applied to the scientific data and associated metadata to enable interoperability of datasets and resources, and provide the name(s) of the data standards that will be applied and describe how these data standards will be applied to the scientific data generated by the research proposed in this project.  If applicable, indicate that no consensus standards exist.
All data will be shared using standard formats: e.g., TIFF files for microscopy imaging, text files with appropriate extensions for computer source code, native filetypes associated with GROMACS.

\subsection*{Element 4: Data Preservation, Access, and Associated Timelines}

\subsubsection*{A. Repository where scientific data and metadata will be archived:}
% Provide the name of the repository(ies) where scientific data and metadata arising from the project will be archived; see Selecting a Data Repository).
\lipsum[1-3]

\subsubsection*{B. How scientific data will be findable and identifiable:}
% Describe how the scientific data will be findable and identifiable, i.e., via a persistent unique identifier or other standard indexing tools.
\lipsum[1-3]

\subsubsection*{C. When and how long the scientific data will be made available:}
% Describe when the scientific data will be made available to other users (i.e., no later than time of an associated publication or end of the performance period, whichever comes first) and for how long data will be available. 
\lipsum[1-3]

\subsection*{Element 5: Access, Distribution, or Reuse Considerations}

\subsubsection*{A. Factors affecting subsequent access, distribution, or reuse of scientific data:}
% NIH expects that in drafting Plans, researchers maximize the appropriate sharing of scientific data. Describe and justify any applicable factors or data use limitations affecting subsequent access, distribution, or reuse of scientific data related to informed consent, privacy and confidentiality protections, and any other considerations that may limit the extent of data sharing. See Frequently Asked Questions for examples of justifiable reasons for limiting sharing of data.
\lipsum[1-3]

\subsubsection*{B. Whether access to scientific data will be controlled:}
% State whether access to the scientific data will be controlled (i.e., made available by a data repository only after approval).
\lipsum[1-3]

\subsubsection*{C. Protections for privacy, rights, and confidentiality of human research participants:}
% If generating scientific data derived from humans, describe how the privacy, rights, and confidentiality of human research participants will be protected (e.g., through de-identification, Certificates of Confidentiality, and other protective measures).  
\lipsum[1-3]

\subsection*{Element 6: Oversight of Data Management and Sharing:}
% Describe how compliance with this Plan will be monitored and managed, frequency of oversight, and by whom at your institution (e.g., titles, roles).

\lipsum[1-3]
\end{document}
